\documentclass[twocolumn,titlepage]{jsarticle}

\setlength{\oddsidemargin}{-1cm}
\setlength{\textwidth}{18cm}
\setlength{\textheight}{27cm}

\setlength{\topmargin}{0pt}
\iftombow
  \addtolength{\topmargin}{-1in}
\else
  \addtolength{\topmargin}{-1truein}
\fi

\title{雪印乳業の乳製品による集団食中毒事件}
\author{[学籍番号] 1211076 [氏名] 佐土瀬 寛}
\date{2015/11/25 課題番号E}
\begin{document}
\maketitle
\section{事故・事件の背景と経過}
\subsection{背景}
失敗知識データベース\cite{database}より引用する。

\begin{quote}
「乳業業界は衛生界のトップランナー」雪印乳業の集団食中毒事件が起きるまで、業界はこう自負していた。1998年1月、
厚生省の定めるHACCPの基準「総合衛生管理製造過程」の承認を最初に取得したのが、雪印乳業など乳業大手であった。
HACCP取得のため、乳業各社は一斉に全製造工程の作業手順をマニュアル化した。最終的に作成した書類は一工場当たり厚さ七、八センチのA4サイズのファイル二冊分に及んだ。
ち密なマニュアルが「最も腐りやすい原料(牛乳)を使う企業が、最も衛生的」との神話を生んだ。
食品業界ではHACCPやISOなどを取得するためのマニュアルを作ればよいという風潮が広がっていた。しかし、マニュアルはあくまでマニュアルに過ぎない。
これを使いこなして作業工程の管理にいかさなければ意味がない。一方で、マニュアル過信では想定外の事態に対応できない。雪印は工場のバルブの洗浄などをマニュアル通りに
作業していなかった。脱脂粉乳から毒素が見付かった大樹工場(北海道大樹町)では、停電というマニュアルの想定外の事態が発生した時、何の対応も取れなかった。
その結果集団食中毒事件となった。
食品業界では価格競争に勝つために正社員に代わりパート、アルバイトの数を増やしたり、下請けに作業を丸投げするケースが増えている。
メーカーは経験の少ないアルバイトや下請けでも一定の品質を保てるように、作業のマニュアル化を進めているが、きちんといかされているとは言い難い。
\end{quote}

\subsection{経過}
専門書\cite{rakujitu}、「知識データベース」を参考に述べる。
\begin{enumerate}
 \item 2000年6月25日大阪市天王寺及び和歌山県内で雪印乳業の低脂肪乳を飲んだ子供が嘔吐や下痢を訴えた。
 \item 6月27日「食中毒症状が出ている」と病院から大阪市保健所へ連絡が入る。
 \item 6月27日に、雪印乳業関西品質保証センターに消費者から大阪工場製造の低脂肪乳を飲用後、数時間して下痢、王との症状が出たという連絡が入る。
 \item 6月28日、雪印乳業は札幌市で株主総会を開く。
 \item 同日、立て続けに3件の低脂肪乳による食中毒の訴えの連絡が大阪市に入る。
 \item 会社側ではその日に、西日本社にて緊急品質管理委員会が開催され、上記3件の苦情情報が確認、集約されたとしている。
 \item 大阪市は、製造自粛、回収、事実の公表を指導。会社側では社内調査を開始、該当する製品に使用した原材料の微生物検査を実施、該当する工場の
 汚染状況を調査した。
 \item 厚生省は、6月30日に大阪市に職員を派遣して関係府県市担当者会議を開催し、同工場が総合衛生管理製造過程の承認施設であったため、
 7月1日に大阪市と合同で立入検査を行った。
 \item 7月2日、大阪府立公衆衛生研究所が「低脂肪乳」から黄色ブドウ球菌のエンテロトキシンA型を検出した。大阪市はこれを病因物質とする食中毒と断定し、
 大阪工場を営業禁止とした。また大阪府警が業務上過失傷害の疑いで捜査を開始。5日 被害者が1万人を超える。6日 雪印乳業の石川哲郎社長(当時)が引責辞任を表明。
 \item 7月10日、大阪市は、有症者の調査、大阪工場への立入検査等の結果に基づき、中間報告をとりまとめ、公表した。報告があった有症者数は14,780名に達した。
 \item 7月11日 雪印が全国21工場の市乳生産を停止。
 \item 7月25日、厚生省が京都、神戸など10工場の操業再開認める。8月2日、厚生相が直営20工場に安全宣言。
 \item 8月18日、大阪市は、大阪工場の製品による集団食中毒事件で、同社大樹(たいき)工場(北海道大樹町)が製造した脱脂粉乳(4月10日製造)から黄色ブドウ球菌の毒素(エンテロトキシンA型)
 が検出されたと発表した。雪印乳製品の販売停止広がる。
 \item 北海道は、大阪市の調査依頼及び厚生省の指示を受けて、8月19日から同工場の調査を行い、8月23日に当該脱脂粉乳の製造に関連した停電の発生、
    生菌数に係る基準に違反する脱脂粉乳の使用、4月1日及び4月10日製造の脱脂粉乳の保存サンプルからエンテロトキシンA型の検出等の調査結果について公表した。
 \item 9月23日、大樹工場から提出された停電事故対策を含む改善計画書を受理し、10月13日に営業禁止命令を解除し、10月14日から操業が再開された。
 \item 9月26日 雪印乳業が来年3月期の業績見通しを発表。連結ベースでの経常損益は538億円の赤字に転落。大阪工場の閉鎖も発表。
 \item 12月20日、雪印乳業の製品による食中毒事件の「厚生省・大阪市原因究明合同専門家会議」は、会議を開き、最終報告をまとめた。食中毒の原因を
 大樹工場(北海道大樹町)製脱脂粉乳と断定。同工場で起こった停電の際に、クリーム分離工程か濃縮工程の回収乳タンクのいずれかで、黄色ブドウ球菌の毒素が発生したとした。
 \item 12月22日、雪印乳業は最終報告書を発表した。報告書は食中毒の原因を大樹工場(北海道大樹町)の脱脂粉乳製造過程と断定、
  3月の停電で大樹工場内の温度管理が不適切になったのが毒素発生の原因とした。
\end{enumerate}

\seection{事故・事件に関与した技術者・技能者が取った行動}

\newpage
\begin{thebibliography}{2}
 \bibitem{database}「失敗知識データベース」 URL:http://www.sozogaku.com/fkd/cf/CA0000622.html
 \bibitem{rakujitu}「雪印の落日-食中毒事件と牛肉偽装事件-」藤原邦達著、緑風出版
\end{thebibliography}


\end{document}
